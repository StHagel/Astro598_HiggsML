\section{Closing remarks and outlook}\label{sec:outlook}

We have seen that even a simple model, which has been trained without any knowledge about high energy particle physics or relativistic kinematics, can perform quite well in classifying collision events.
We have also seen that only a few steps of optimization can bring up the performance of the neural network significantly.
Eventhough we get some nice results, the model is far from perfect and does not compare to the ones actually used for the challenge or research done at CERN.
But even within the scope of this project there are still some ways to improve on the project.
One idea, which I did not manage to implement yet is to improve the imputer for the missing Higgs candidate mass data by training a separate regression model on the events with valid data to predict a value for the missing data.
Another way to improve is to implement some of the statistical tweaks mentioned by Balázs Kégl in his talk, in which he introduces the challenge.\footnote{http://opendata.cern.ch/record/330}

